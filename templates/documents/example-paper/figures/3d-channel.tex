%!TEX root = 3d-channel.tex
\documentclass{standalone}

\usepackage{amsmath}
\usepackage{amsfonts}
\usepackage{amssymb}
\usepackage{amsthm}
\usepackage{bbm}
\usepackage{tikz}
\usepackage{pgfplots}
\usepackage{pgfplotstable}
\usepackage{rotating}

\usepgfplotslibrary{groupplots}

\usetikzlibrary{arrows}
\usetikzlibrary{shapes}
\usetikzlibrary{decorations.text}
\usetikzlibrary{quantikz}
\usetikzlibrary{arrows.meta}
\usetikzlibrary{quotes}
\usetikzlibrary{calc}
\usetikzlibrary{fadings}
\usetikzlibrary{decorations.pathreplacing}
\usetikzlibrary{patterns}
\usetikzlibrary{3d}
\usetikzlibrary{intersections}
\usetikzlibrary{bending}
\usetikzlibrary{cd}
\usetikzlibrary{fit}
\usetikzlibrary{decorations.markings}

\usepackage{siunitx}

\textwidth = 6.5 in
\textheight = 8 in

\usepackage{bm}
\newcommand{\ve}[1]{\bm{#1}}

\newcommand{\ba}{\ve{a}}
\newcommand{\bb}{\ve{b}}
\newcommand{\bu}{\ve{u}}
\newcommand{\bv}{\ve{v}}
\newcommand{\br}{\ve{r}}
\newcommand{\bp}{\ve{p}}
\newcommand{\bt}{\ve{t}}
\newcommand{\bn}{\ve{n}}
\newcommand{\bc}{\ve{c}}
\newcommand{\bq}{\ve{q}}
\newcommand{\bff}{\ve{f}}
\newcommand{\bw}{\ve{w}}
\newcommand{\by}{\ve{y}}
\newcommand{\bx}{\ve{x}}
\newcommand{\be}{\ve{e}}
\newcommand{\bg}{\ve{g}}
\newcommand{\bh}{\ve{h}}
\newcommand{\bs}{\ve{s}}
\newcommand{\bk}{\ve{k}}

\newcommand{\bA}{\ve{A}}
\newcommand{\bD}{\ve{D}}
\newcommand{\bJ}{\ve{J}}
\newcommand{\bS}{\ve{S}}
\newcommand{\bB}{\ve{B}}
\newcommand{\bU}{\ve{U}}
\newcommand{\bV}{\ve{V}}
\newcommand{\bW}{\ve{W}}
\newcommand{\bE}{\ve{E}}
\newcommand{\bQ}{\ve{Q}}
\newcommand{\bP}{\ve{P}}
\newcommand{\bGG}{\ve{G}}
\newcommand{\bRR}{\ve{R}}
\newcommand{\bX}{\ve{X}}
\newcommand{\bM}{\ve{M}}
\newcommand{\bC}{\ve{C}}
\newcommand{\bF}{\ve{F}}
\newcommand{\bI}{\ve{I}}
\newcommand{\bL}{\ve{L}}
\newcommand{\bR}{\ve{R}}

\newcommand{\tn}{\tilde{n}}

\newcommand{\cM}{\mathcal{M}}

\newcommand{\dd}{\text{d}}
\newcommand{\ddd}{\dd R \dd \dot{R}}
\newcommand{\bbE}{\mathbb{E}}
\newcommand{\Rdot}{\dot{R}}
\newcommand{\Rddot}{\ddot{R}}

\newcommand{\muR}{\mu_R}
\newcommand{\muRdot}{\mu_{\dot{R}}}
\newcommand{\sigmaR}{\sigma_R}
\newcommand{\sigmaRdot}{\sigma_{\dot{R}}}

\newcommand{\bmu}{\vec{\boldsymbol{\mu}}}
\newcommand{\btheta}{\vec{\boldsymbol{\theta}}}

\newcommand{\brho}{\boldsymbol{\rho}}
\newcommand{\bLambda}{\boldsymbol{\Lambda}}
\newcommand{\blambda}{\boldsymbol{\lambda}}
\newcommand{\bSigma}{\boldsymbol{\Sigma}}
\newcommand{\bEps}{\boldsymbol{\varepsilon}}
\newcommand{\beps}{\boldsymbol{\varepsilon}}
\newcommand{\bxi}{\boldsymbol{\xi}}
\newcommand{\balpha}{\boldsymbol{\alpha}}
\newcommand{\heps}{\hat{\varepsilon}}
\newcommand{\bdelta}{\boldsymbol{\delta}}
\newcommand{\vblambda}{\vec{\boldsymbol{\lambda}}}
\newcommand{\vbsigma}{\vec{\boldsymbol{\sigma}}}
\newcommand{\vbEps}{\vec{\boldsymbol{\varepsilon}}}
\newcommand{\vbdelta}{\vec{\boldsymbol{\delta}}}
\newcommand{\eps}{\varepsilon}

\newcommand{\vq}{\vec{q}}

\newcommand{\vba}{\vec{\ve{a}}}
\newcommand{\vbu}{\vec{\ve{u}}}
\newcommand{\vbx}{\vec{\ve{x}}}
\newcommand{\vby}{\vec{\ve{y}}}
\newcommand{\vbm}{\vec{\ve{M}}}
\newcommand{\vbmm}{\vec{\ve{m}}}
\newcommand{\vbv}{\vec{\ve{v}}}
\newcommand{\vbh}{\vec{\ve{h}}}
\newcommand{\vbbh}{\vec{\bar{\ve{h}}}}
\newcommand{\vbs}{\vec{\ve{s}}}
\newcommand{\vbF}{\vec{\ve{F}}}
\newcommand{\vbg}{\vec{\ve{g}}}
\newcommand{\vbq}{\vec{\ve{q}}}
\newcommand{\vbr}{\vec{\ve{r}}}

\newcommand{\hx}{\hat{x}}
\newcommand{\hy}{\hat{y}}
\newcommand{\hz}{\hat{z}}

\newcommand{\hbx}{\hat{\ve{x}}}
\newcommand{\hby}{\hat{\ve{y}}}
\newcommand{\hbz}{\hat{\ve{z}}}

\newcommand{\hbF}{\widehat{\ve{F}}}
\newcommand{\hbq}{\widehat{\ve{q}}}
\newcommand{\hbg}{\widehat{\ve{g}}}
\newcommand{\hw}{\widehat{w}}

\newcommand{\hR}{\widehat{R}}
\newcommand{\hRdot}{\widehat{\dot{R}}}

\newcommand{\trho}{\widetilde{\rho}}
\newcommand{\tbrho}{\widetilde{\boldsymbol{\rho}}}
\newcommand{\tx}{\widetilde{x}}
\newcommand{\tbx}{\widetilde{\ve{x}}}
\newcommand{\tbB}{\widetilde{\ve{B}}}

\newcommand{\mom}{\mu}
\newcommand{\bmom}{\boldsymbol{\mu}}
\newcommand{\vbmom}{\vec{\bmom}}
\newcommand{\hbmom}{\widehat{\bmom}}


\newcommand{\EV}{\mathbb{E}}

\newcommand{\dt}{\Delta \tau}
\newcommand{\gdot}{\dot\gamma}

\newcommand\Rey{\mbox{\text{Re}}\xspace}

\newcommand{\cD}{\mathcal{D}}
\newcommand{\cI}{\mathcal{I}}
\newcommand{\cG}{\mathcal{G}}
\newcommand{\cC}{\mathcal{C}}
\newcommand{\cL}{\mathcal{L}}
\newcommand{\dL}{\mathbf{L}}
\newcommand{\dW}{\mathbf{W}}
\newcommand{\dO}{\mathbf{O}}
\newcommand{\dM}{\mathbf{M}}
\newcommand{\dU}{\mathbf{U}}
\newcommand{\dD}{\mathbf{D}}

\newcommand{\cbL}{\overline{\mathcal{L}}}
\newcommand{\baru}{\bar{u}}
\newcommand{\barc}{\bar{c}}
\newcommand{\bars}{\bar{s}}

\newcommand{\bbarf}{\bar{\ve{f}}}
\newcommand{\barbu}{\bar{\ve{u}}}

\definecolor{lightblue}{rgb}{0.63, 0.74, 0.78}
\definecolor{seagreen}{rgb}{0.18, 0.42, 0.41}
\definecolor{orange}{rgb}{0.85, 0.55, 0.13}
\definecolor{silver}{rgb}{0.69, 0.67, 0.66}
\definecolor{rust}{rgb}{0.72, 0.26, 0.06}
\definecolor{purp}{RGB}{68, 14, 156}

\colorlet{lightrust}{rust!50!white}
\colorlet{lightorange}{orange!25!white}
\colorlet{lightlightblue}{lightblue}
\colorlet{lightsilver}{silver!30!white}
\colorlet{darkorange}{orange!75!black}
\colorlet{darksilver}{silver!65!black}
\colorlet{darklightblue}{lightblue!65!black}
\colorlet{darkrust}{rust!85!black}
\colorlet{darkseagreen}{seagreen!85!black}




\definecolor{RYB1}{RGB}{207, 37, 37}
\definecolor{RYB2}{RGB}{37, 91, 207}
\definecolor{RYB3}{RGB}{37, 207, 91}
\definecolor{RYB4}{RGB}{163,26,145}
\definecolor{RYB5}{RGB}{253, 180, 98}
\definecolor{RYB6}{RGB}{179, 222, 105}
\definecolor{RYB7}{RGB}{128, 177, 211}

\pgfplotscreateplotcyclelist{newcolors}{
    {RYB1,every mark/.append style={fill=RYB1,mark size={1.5}},mark=*},
    {RYB2,every mark/.append style={fill=RYB2},mark=square*},
    {RYB3,every mark/.append style={fill=RYB3,mark size={3}},mark=triangle*},
    {RYB4,every mark/.append style={fill=RYB4,mark size={3}},mark=diamond*},
    {RYB5,every mark/.append style={fill=RYB5,mark size={3}},mark=pentagon*},
    {RYB6,every mark/.append style={fill=RYB6,mark size={4}},mark=10-pointed star},
    {RYB7,every mark/.append style={fill=RYB7},mark=*},
}

\pgfplotsset{
    standard/.style={
    compat=1.18,
    scale only axis,
    width=0.5\textwidth,
    enlarge x limits=0.05,
    enlarge y limits=0.05,
    max space between ticks=30,
	cycle list name=newcolors,
	}
}

\newcounter{randarcs}

\definecolor{RYB1}{RGB}{207, 37, 37}
\definecolor{RYB2}{RGB}{37, 91, 207}
\definecolor{RYB3}{RGB}{37, 207, 91}
\definecolor{RYB4}{RGB}{163,26,145}
\definecolor{RYB5}{RGB}{253, 180, 98}
\definecolor{RYB6}{RGB}{179, 222, 105}
\definecolor{RYB7}{RGB}{128, 177, 211}

\usepgfplotslibrary{colormaps}%

\begin{document}
\pagestyle{empty}

% Streamwise -> 2pi
% Spanwise -> Pi
% Wallnormal -> 2

% dir 1: depth, -> x = streamwise = len 6
% dir 2: width  -> y = wallnormal = len 2
% dir 2: height -> z = spanwise   = len 3

\begin{tikzpicture}[randarc/.style={out=\angA+90+\angB/6,in=\angA+90-\angB/6,looseness=\lsns}]

    \newcommand{\Depth}{4.5}
    \newcommand{\Height}{3}
    \newcommand{\Width}{2}

    \newcommand{\myHeight}{3}
    \newcommand{\myWidth}{2}

    \newcommand{\off}{4}

    %main box
    \coordinate (O) at (0,0,0);
    \coordinate (A) at (0,\Width,0);
    \coordinate (B) at (0,\Width,\Height);
    \coordinate (C) at (0,0,\Height);
    \coordinate (D) at (\Depth,0,0);
    \coordinate (E) at (\Depth,\Width,0);
    \coordinate (F) at (\Depth,\Width,\Height);
    \coordinate (G) at (\Depth,0,\Height);
    \draw[thick,black] (O) -- (A) -- (E) -- (D) -- cycle;% Back Face
    \draw[thick,black] (O) -- (A) -- (B) -- (C) -- cycle;% Left Face
    \draw[thick,black] (D) -- (E) -- (F) -- (G) -- cycle;% Right Face
    \draw[thick,black] (C) -- (B) -- (F) -- (G) -- cycle;% Front Face
    \draw[thick,black] (A) -- (B) -- (F) -- (E) -- cycle;% Top Face

    % Parabola vert
    \draw[rust, -latex] (0,0.33,0.5*\Height) -- (0.55,0.33,0.5*\Height);
    \draw[rust, -latex] (0,0.66,0.5*\Height) -- (0.67,0.66,0.5*\Height);
    \draw[rust, -latex] (0,1+0.33,0.5*\Height) -- (0.67,1+0.33,0.5*\Height);
    \draw[rust, -latex] (0,1+0.66,0.5*\Height) -- (0.55,1+0.66,0.5*\Height);
    \draw[rust, -latex] (0,0.5*\Width,1.5) -- (0.73,0.5*\Width,1.5);
    \draw[rust, thick, fill=black, fill opacity=0.1] (0,0,0.5*\Height) to[in=0,out=0,looseness=1.22] (0,\Width,0.5*\Height) to (0,0,0.5*\Height) -- cycle;

    \node[anchor=east] at (0,0.5*\Width,\Height) {$x_2$};
    \node[anchor=north] at (0.5*\Depth,0,\Height) {$x_1$};
    \node[anchor=north west] at (\Depth,0,0.5*\Height) {$x_3$};

    \node[anchor=north,inner sep=5pt] at (0.5*\Depth,\Width,0) {\small Turbulent Channel Flow};

    % Averaging Plane
    \coordinate (O) at (0     ,0.2*\Width,0);
    \coordinate (A) at (\Depth,0.2*\Width,0);
    \coordinate (B) at (\Depth,0.2*\Width,\myHeight);
    \coordinate (C) at (0     ,0.2*\Width,\myHeight);
    \draw[black,very thick,fill=silver,fill opacity=0.3] (O) -- (A) -- (B) -- (C) -- cycle;

    \pgfmathsetseed{221}
    \foreach \X in {1,...,60}
    {
    \pgfmathsetmacro{\myx}{2.25+4*(rnd-0.5)}
    \pgfmathsetmacro{\myy}{1.5+1.9*(rnd-0.5)}
    \pgfmathsetmacro{\angA}{360*rnd}
    \pgfmathsetmacro{\radA}{0.15+0.15*rnd}
    \pgfmathsetmacro{\myxp}{\myx+\radA*cos(\angA)}
    \pgfmathsetmacro{\myyp}{\myy+\radA*sin(\angA)}
    \pgfmathsetmacro{\angB}{360*sin(90*rnd)}
    \pgfmathsetmacro{\lsns}{3+0.1*\radA}
    \pgfmathsetmacro{\spencerand}{0.2+0.8*rnd}
    \draw[-, seagreen, very thick, opacity=\spencerand] 
        (\myx,0.2*\Width,\myy) to[randarc] (\myxp,0.2*\Width,\myyp);
    }

    \begin{scope}[xshift=6.25cm,yshift=-1cm]

        \newcommand{\boxlenx}{3.3}
        \newcommand{\boxleny}{2.2}

        \node (b) at (0,0.65*\boxleny,0) {};
        \node (d) at (\boxlenx*1.3,0.65*\boxleny,0) {};

        % Draw some slices
        \foreach \Y in {4,3,...,0}
        {
        \newcommand{\shift}{\Y*0.2}
        \draw[fill=silver!30,very thick]  (0+\shift,0+\shift,0) -- (\boxlenx+\shift,0+\shift,0) -- (\boxlenx+\shift,\boxleny+\shift,0) -- (0+\shift,\boxleny+\shift,0) -- cycle;
        }
        \draw[very thick,-latex] (\boxlenx+0.6,0,0) -- node[midway,anchor=north west,inner sep=1pt,align=center] {\small Time \\ \small (Avg.)} (\boxlenx+1.3,0.8,0);
        \node[anchor=north] at (0.5*\boxlenx,0,0) {$x_1$};
        \node[anchor=east] at (0,0.5*\boxleny,0) {$x_3$};

        % Draw swirls
        \pgfmathsetseed{200}
        \foreach \X in {1,...,100}
        {
        \pgfmathsetmacro{\myx}{1.65+2.5*(rnd-0.5)}
        \pgfmathsetmacro{\myy}{1.10+1.5*(rnd-0.5)}
        \pgfmathsetmacro{\angA}{360*rnd}
        \pgfmathsetmacro{\radA}{0.15+0.15*rnd}
        \pgfmathsetmacro{\myxp}{\myx+\radA*cos(\angA)}
        \pgfmathsetmacro{\myyp}{\myy+\radA*sin(\angA)}
        \pgfmathsetmacro{\angB}{360*sin(90*rnd)}
        \pgfmathsetmacro{\lsns}{3+0.1*\radA}
        \pgfmathsetmacro{\spencerand}{0.2+0.8*rnd}
        \path[-{Latex},name path=test-arc] 
        (\myx,\myy) to[randarc] (\myxp,\myyp) ;
        \def\HasIntersection{0}
        \ifnum\X>1
         \foreach \Y in {1,...,\number\value{randarcs}}
         {\path[name intersections={of=\Y-arc and test-arc,total=\t},
         /utils/exec=\ifnum\t>0
          \xdef\HasIntersection{1}%\typeout{intersects}
         \fi];
         }
        \fi
        \ifnum\HasIntersection=0
            \stepcounter{randarcs}
            \draw[-, seagreen, very thick, opacity=\spencerand] 
                (\myx,\myy,0) to[randarc] (\myxp,\myyp,0);
            \path[name path global=\number\value{randarcs}-arc]
               (\myx,\myy,0) to[randarc] (\myxp,\myyp,0)-- cycle;
        \fi
        }
    \end{scope}

    \draw[very thick,->,looseness=0.9] (1.03*\Depth,0.24*\Width,0*\Height) to[in=180-45,out=45] node[midway,anchor=south,align=center]{\small Plane \\ \small (Avg.)} (b);

    \begin{scope}[xshift=12.25cm,yshift=-0.75cm]
        \node (c) at (-0.1,1.2,0) {};
        \begin{axis}[
            width=4cm,
            height=4cm,
            xticklabels=\empty,
            yticklabels=\empty,
            xlabel=$x_2$,
            xlabel near ticks,
            ]
            \addplot+[very thick,no marks,rust,smooth] table[col sep=comma] {../data/turbulent_channel/reconstructions/D1121_dns.csv};
            \node[align=center] at (axis cs:0,6) {
                \footnotesize Reynolds \\ \footnotesize Stress};
        \end{axis} 
    \end{scope}

    \draw[very thick,->,looseness=0.9] (d) to[in=180-45,out=45] node[midway,anchor=south,align=center]{\small Recover \\ \small Operator} (c);

\end{tikzpicture}

\end{document}
